\documentclass[11pt]{article}

\usepackage{times}
\usepackage[english]{babel}

% -----------------------------------------------
% especially use this for you code
% -----------------------------------------------

\usepackage{courier}
\usepackage{listings}
\usepackage{color}
\usepackage{tabularx}
\usepackage{graphicx}

\definecolor{Gray}{gray}{0.95}

\definecolor{mygreen}{rgb}{0,0.6,0}
\definecolor{mygray}{rgb}{0.5,0.5,0.5}
\definecolor{mymauve}{rgb}{0.58,0,0.82}

\lstset{language=C++,
	basicstyle = \normalsize\ttfamily,   % the size and fonts that are used
	tabsize = 2,                    % sets default tabsize
	breaklines = true,              % sets automatic line breaking
	keywordstyle=\color{blue}\ttfamily,
	stringstyle=\color{red}\ttfamily,
	commentstyle=\color{mygreen}\ttfamily,
	numbers=left,
	keepspaces=true,
	showspaces=false,
	showstringspaces=false,
}

\begin{document}

\title{Programming in C/C++ \\
       Exercises set two: inheritance
}
\date{\today}
\author{Christiaan Steenkist \\
Jaime Betancor Valado \\
Remco Bos \\
}

\maketitle
\section*{Exercise 9, addition/subtraction hierarchy header files}
These are the final header files for this class hierarchy.

\subsection*{Code listings}
\lstinputlisting[caption = addition.ih]{src/a11/addition/addition.ih}
\lstinputlisting[caption = addition.h]{src/a11/addition/addition.h}
\lstinputlisting[caption = binops.ih]{src/a11/binops/binops.ih}
\lstinputlisting[caption = binops.h]{src/a11/binops/binops.h}
\lstinputlisting[caption = subtraction.ih]{src/a11/subtraction/subtraction.ih}
\lstinputlisting[caption = subtraction.h]{src/a11/subtraction/subtraction.h}

\section*{Exercise 10, compound assignment operators}
Here we implement the compound assignment operators in the class hierarchy.

\subsection*{Code listings}
\lstinputlisting[caption = operator\texttt{+=}.cc]{src/a11/addition/operator+=.cc}
\lstinputlisting[caption = operator\texttt{-=}.cc]{src/a11/subtraction/operator-=.cc}

\section*{Exercise 11, free binary operators}
Here we implement the free binary operators in the class hierarchy as well as the add and sub members.

\subsection*{Code listings}
\lstinputlisting[caption = main.cc]{src/a11/main.cc}
\lstinputlisting[caption = operator+.cc]{src/a11/addition/operator+.cc}
\lstinputlisting[caption = operator-.cc]{src/a11/subtraction/operator-.cc}

\subsubsection*{Operation code}
\lstinputlisting[caption = operations.ih]{src/a11/operations/operations.ih}
\lstinputlisting[caption = operations.h]{src/a11/operations/operations.h}
\lstinputlisting[caption = add.cc]{src/a11/operations/add.cc}
\lstinputlisting[caption = sub.cc]{src/a11/operations/sub.cc}

\section*{Exercise 12, matrix copying}
We were tasked with initializing an array with copies of a matrix \texttt{mat} with only a \texttt{new} statement.

\subsection*{Solution}
Because the default constructor is always called we made a new class that is derived from matrix.
This \texttt{Copymatrix} holds a static \texttt{Matrix} that is used to initialize the class.
By returning a \texttt{Matrix} pointer the unnecessary data is sliced off.

\subsection*{Code listings}
\lstinputlisting[caption = main.ih]{src/a12/main.ih}
\lstinputlisting[caption = main.h]{src/a12/main.h}
\lstinputlisting[caption = factory.cc]{src/a12/factory.cc}
\lstinputlisting[caption = main.cc]{src/a12/main.cc}

\subsubsection*{Copymatrix}
\lstinputlisting[caption = copymatrix.ih]{src/a12/copymatrix.ih}
\lstinputlisting[caption = copymatrix.h]{src/a12/copymatrix.h}
\lstinputlisting[caption = copymatrix1.cc]{src/a12/copymatrix1.cc}

\section*{Exercise 13, red thread}
We were tasked with using inheritence to include the \texttt{Limits} class into a number of other classes.

\subsection*{Code listings}
\lstinputlisting[caption = fighter.h]{src/a13/fighter/fighter.h}
\lstinputlisting[caption = fighter's altitudeto.cc]{src/a13/fighter/altitudeto.cc}
\lstinputlisting[caption = monitor.h]{src/a13/monitor/monitor.h}
\lstinputlisting[caption = time.h]{src/a13/time/time.h}

\end{document}
